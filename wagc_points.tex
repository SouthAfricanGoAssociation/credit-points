\documentclass{article}

\usepackage[german]{babel}

\title{How to use \texttt{wagc\_points.py}}
\author{Reinhardt Messerschmidt}
\date{2006-10-22}

\begin{document}

\maketitle

\section{Introduction}

\subsection{History}
The calculation of SAGA ranks and WAGC points used to be done with a Visual Basic program maintained by Paul Edwards. In September 2005,
SAGA officially recognised the ranks calculated by the script run on the SAGC website after its algorithm was adjusted to be the
same as the Visual Basic program's algorithm. However, the script did not calculate WAGC points, and so the  Python script 
\texttt{wagc\_points.py} was written to fill this gap.

\subsection{Objective}
See the \emph{WAGC points system} page on the SAGC wiki for what \texttt{wagc\_points.py} tries to do.

\section{Input}
\texttt{wagc\_points.py} requires two files as input: a parameter file and a data file.

\subsection{Parameter file}
The parameter file should have the following format:

\noindent\\
\emph{\flq Annual depreciation factor\frq}\\
$*$\\ 
\emph{\flq Representative's depreciation cap\frq}\\
$*$\\
\emph{\flq Membership points\frq}\\
$*$\\
\emph{\flq SA Championship points for 1st place\frq}\\
\emph{\flq SA Championship points for 2nd place\frq}\\
\emph{\flq \ldots\frq}\\
\emph{\flq SA Championship points for the rest\frq}\\
$*$\\
\emph{\flq Number of Internet Championship tournaments taken into account\frq}\\
\emph{\flq Internet Championship points for 1st place\frq}\\
\emph{\flq Internet Championship points for 2nd place\frq}\\
\emph{\flq \ldots\frq}\\
\emph{\flq Internet Championship points for the rest\frq}\\
$*$\\
\emph{\flq Participation points cap\frq}\\
\emph{\flq Participation points cap per opponent\frq}\\
\emph{\flq Participation points for a tournament game\frq}\\
\emph{\flq Participation points for a club game\frq}\\
\emph{\flq Participation points for an internet game\frq}\\

Note that the lengths of the lists of SA Championship and Internet Championship points are not fixed --- the lists can be made longer or 
shorter as needed.

\subsection{Data file}

The data file should have the following format:

\noindent\\
\emph{\flq Calender year for which WAGC points are being calculated\frq}\\
\emph{\flq Opening balances file\frq}\\
\emph{\flq Representatives file\frq}\\
\emph{\flq Membership file\frq}\\
\emph{\flq SA Championship file\frq}\\
\emph{\flq Internet Championship file\frq}\\
\emph{\flq Games file\frq}

\subsubsection{Opening balances file}
\texttt{wagc\_points.py}'s output from the previous calendar year.

\subsubsection{Representatives file}
The representatives file should have a line for each representative in the following format:

\noindent\\
\emph{\flq Name\frq}\texttt{,}
\emph{\flq Sponsorship level\frq}\texttt{,}
\emph{\flq Number of times\frq}\texttt{,}
\emph{\flq Position\frq}\texttt{,}
\emph{\flq Number of participants\frq}

Sponsorship level is a number between 0 and 100, based on the level of sponsorship of the tournament.  Also, the code assumes no
player appears more than once in the representatives file.

\subsubsection{Membership file}
The members for the current calendar year, one name per line.

\subsubsection{SA Championship file}
The SA Championship file should have the following format:

\noindent\\
\emph{\flq Name of 1st placed qualifier\frq}[\texttt{,}\emph{\flq Name of 1st placed qualifier\frq}[\texttt{,}\ldots]]\\
\emph{\flq Name of 2nd placed qualifier\frq}[\texttt{,}\emph{\flq Name of 2nd placed qualifier\frq}[\texttt{,}\ldots]]\\
\emph{\flq \ldots\frq}\\
$*$\\
\emph{\flq Names of participants in SA Open and qualifying tournaments, one name per line\frq}\\

Square brackets indicate optional arguments that are only used if there is a tie for a position. It is not necessary to 
exclude the names of qualifiers from the names below the $*$, or to have only one instance of a name below the $*$ --- 
\texttt{wagc\_points.py} handles any repetition.

Note that this list should only contain SAGA members.  Otherwise, points are allocated (but not awarded) to players who are not SAGA members,
disadvantaging SAGA members placed below them.

\subsubsection{Internet Championship file}
The Internet Championship file should have a group of lines for each internet tournament, in the following format:

\noindent\\
\emph{\flq Name of 1st placed player\frq}[\texttt{,}\emph{\flq Name of 1st placed player\frq}[\texttt{,}\ldots]]\\
\emph{\flq Name of 2nd placed player\frq}[\texttt{,}\emph{\flq Name of 2nd placed player\frq}[\texttt{,}\ldots]]\\
\emph{\flq \ldots\frq}\\
$*$\\

Square brackets indicate optional arguments that are only used if there is a tie for a position.

Note that this list should only contain SAGA members.  Otherwise, points are allocated (but not awarded) to players who are not SAGA members,
disadvantaging SAGA members placed below them.

\subsubsection{Games file}
The SAGC ranking system script generates a file with information on all the logged games. Executing the command

\noindent\\
\texttt{python make\_input.py}\emph{\ \flq Calender year for which WAGC points are being calculated\frq\ \flq SAGC games file\frq}\\

creates a file called \texttt{games.txt}, which should be used as the games file. \texttt{make\_input.py} does the following to the 
SAGC games file:
\begin{itemize}
\item[-] Removes all free games and games not in the calendar year for which WAGC points are being calculated.
\item[-] Changes the \emph{date} format from \emph{dd/mm/yyyy} to \emph{yyyymmdd}.
\item[-] Changes the \emph{game type} format from \emph{0.5$\vert$1$\vert$1.5} to \emph{i$\vert$c$\vert$t}.
\item[-] Changes the \emph{result} format from \emph{W$\vert$B} to \emph{w$\vert$b}.
\item[-] Removes the \emph{effective handicap} and \emph{comments} columns.
\item[-] Changes the order of the remaining columns.
\item[-] Sorts the games by date.
\end{itemize}

\section{Execution}

The following command should be used:

\noindent\\
\texttt{python wagc\_points.py}\emph{\ \flq parameter file\frq\ \flq data file\frq}

\section{Output}

\texttt{wagc\_points.py} creates a file called \texttt{wagc\_points.txt} giving a summary of the WAGC points standings. It also 
creates a directory called \texttt{recordsheets} containing a detailed report for each player.

\end{document}
